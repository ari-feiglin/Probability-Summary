We will now generalize our discussion to not just focus on discrete probability spaces but general ones as well.
This will be somewhat brief and we will then focus on continuous probability spaces.
First let's revise our previous definition of a probability space.

\begin{defn*}

	A \ppemph{$\sigma$-algebra} is a set $\cF\subseteq\powsetof\Omega$ for some set $\Omega$ which satisfies the
	following conditions:
	\begin{msecenumerate}
		\mitem $\varnothing\in\cF$
		\mitem If $\set{U_n}_{n\in\bN}\in\cF$ then their union is in $\cF$ as well:
		\[ \bigcup_{n\in\bN}U_n\in\cF \]
		\mitem For every $U\in\cF$, its complement is in $\cF$ as well: $U^c\in\cF$.
	\end{msecenumerate}

\end{defn*}

It then follows that $\Omega\in\cF$ since it is the empty set's complement.
$\cF$ is also closed under finite unions since we can define $U_n=\varnothing$ for every $n$ outside of the indexing
set.
And if $\set{U_n}_{n\in\bN}\in\cF$, then $U_n^c\in\cF$ and therefore
\[ \parens{\bigcup_{n\in\bN}U_n^c}^c\in\cF \implies \bigcap_{n\in\bN}U_n\in\cF \]
So $\cF$ is closed under intersections as well.

\begin{prop*}

	If $I$ is an arbitrary indexing set and $\set{\cF_i}_{i\in I}$ is a series of $\sigma$-algebras, then
	\[ \bigcap_{i\in I}\cF_i \]
	is also a $\sigma$-algebra.

\end{prop*}

\begin{proof}

	We will show that the intersection satisfies the conditions for being a $\sigma$-algebra.

	\begin{msecenumerate}[0pt]
		\mitem Since for every $\cF_i$, $\varnothing\in\cF_i$, $\varnothing$ is in the intersection as well.
		\mitem If $\set{U_n}_{n\in\bN}$ is in the intersection, $\set{U_n}_{n\in\bN}\in\cF_i$ for every $i\in I$, so:
		\[ \bigcup_{n\in\bN}U_n\in\cF_i \]
		Since $\cF_i$ is a $\sigma$-algebra, and since this is true for every $i$, the union of $U_n$ is in the
		intersection of $\cF_i$.
		\mitem If $U$ is in the intersection, it is in every $\cF_i$, and therefore $U^c$ is in every $\cF_i$ as well
		and is therefore in the intersection.
	\end{msecenumerate} 

	\hfill$\qed$

\end{proof}

So if we have a characteristic which we want a $\sigma$-algebra to have, then we can take the minimum $\sigma$-algebra
which has this characteristic (minimum under $\subseteq$) by taking the intersection of all the $\sigma$-algebras which
have this characteristic.
The most famous and useful example of this are Borel Sets:

\begin{defn*}

	Given a set $S\subseteq\bR$, we define the \ppemph{Borel Set} of $S$, $\borelof S$, to be the minimum $\sigma$-algebra
	which contains every closed interval in $S$.

	Furthermore, while this is redundant, we will require $\borelof S\subseteq\powsetof S$.

\end{defn*}

$\borelof S$ exists since we can take the intersection of all the $\sigma$-algebras which contain every closed interval of
$S$ and are subsets of its powerset.
This intersection is non-empty since $\powsetof S$ is contained in it.

\begin{defn*}

	A \ppemph{Probability Space} is a triplet
	\[ \tuple{\Omega, \cF, \prob} \]
	Where $\Omega$ is a set called the \ppemph{sample space}, $\cF$ is a $\sigma$-algebra over $\Omega$, and $\prob$ is a
	probability function over $\cF$, a function
	\[ \prob\colon\cF\longrightarrow[0,1] \]
	Where $\probof\Omega=1$ and if $\set{A_n}_{n\in\bN}$ are disjoint then
	\[ \probof{\bigsqcup_{n\in\bN}A_n} = \sum_{n\in\bN}\probof{A_n} \]

\end{defn*}

\begin{thrm*}

	There \emph{exists} a probability function:
	\[ \prob\colon\borelof{[0,1]}\longrightarrow[0,1] \]
	Where for every $[a,b]\subseteq[0,1]$, $\probof{[a,b]}=b-a$.

\end{thrm*}

Notice that this doesn't tell us much about the actual definition of the Lebesgue measure, just a certain criteria it must
fulfil.
We will not be proving this theorem as we don't have the necessary tools to do so.
To prove this we just need to show that the Lebesgue Measure satisfies the criteria.

\begin{prop*}

	Suppose $\prob$ is the probability function defined over $\borelof{[0,1]}$ discussed above.
	Then

	\begin{msecenumerate}
		\mitem $\probof{\set a}=0$
		\mitem $\probof{(a,b)}=b-a$
		\mitem If $Q$ is countable, then $\probof{Q}=0$.
	\end{msecenumerate}

\end{prop*}

\begin{proof}

	\begin{msecenumerate}[0pt]
		\mitem Notice that in \ppref[theorem]{contProbTheorem}, we did not assume that the probability space was discrete, so
			we will use the result here.
			Let's define $A_n=\bracks{a,a+\frac1n}$ for every $n\in\bN$.
			Then $\set{A_n}_{n\in\bN}$ is decreasing, and its intersection is $\set{a}$.
			Furthermore, notice that $\probof{A_n}=a+\frac1n-a=\frac1n$.
			So:
			\[ \probof{\set a} = \probof{\bigcap_{n\in\bN}A_n} = \lim_{n\to\infty}\probof{A_n} = \lim_{n\to\infty}\frac1n = 0 \]
			As required.

		\mitem Notice that:
			\[ \probof{[a,b]} = \probof{(a,b)\cup\set a\cup\set b} = \probof{(a,b)} + \probof{\set a}+\probof{\set b} = \probof{(a,b)} \]
			Since the probability of a singleton is $0$.
			So:
			\[ \probof{(a,b)} = \probof{[a,b]} = b - a \]

		\mitem We know that
			\[ \probof{Q} = \probof{\bigsqcup_{q\in Q}\set q} \]
			And since this is a disjoint countable union, this is equal to:
			\[ = \sum_{q\in Q}\probof{\set q} = 0 \]
	\end{msecenumerate}

	\hfill$\qed$

\end{proof}

\begin{defn*}

	If $\tuple{\Omega,\cF,\prob}$ is a probability space, a \ppemph{random variable} is a function
	\[ X\colon\Omega\longrightarrow\bR \]
	Where for every $B\in\borelof\bR$, the preimage of $B$, $X^{-1}(B)$, is an event in $\cF$.
	The reason for this is so $X\in B$ (which is the set $\set{\omega\in\Omega}[\omega\in X^{-1}(B)]$) is an event in $\cF$.

	The \ppemph{distribution function} of $X$ is a function
	\[ \prob_X\colon\borelof\bR\longrightarrow[0,1] \]
	Such that $\probof[X]{B}=\probof{X\in B}=\probof{X^{-1}(B)}$.
	Note that $\prob_X$ is a probability function over $\tuple{\bR,\borelof\bR}$.

	Two random variables, $X$ and $Y$, are \ppemph{distributively equal} (or equivalent) if $\prob_X=\prob_Y$.
	This is denoted $X\deq Y$.

\end{defn*}

\begin{defn*}

	If $X$ is a random variable, then the \ppemph{cumulative distribution} of $X$ is the function
	\[ F_X\colon\bR\longrightarrow[0,1] \]
	Defined by
	\[ F_X(s) = \probof{X\leq s} = \probof{X^{-1}\big((-\infty,s]\big)} \]

	The \ppemph{complementary cumulative distribution function} (or \ppemph{tail distribution}) of $X$ is the function
	\[ \ccum\colon\bR\longrightarrow[0,1] \]
	Defined by
	\[ \ccumof s = \probof{X>s} = 1 - \cumof s \]

\end{defn*}

\begin{prop*}

	If $X$ is a random variable, then:

	\begin{msecenumerate}
		\mitem $F_X$ is increasing.
		\mitem $\lim\limits_{x\to\infty} F_X(x)=1$ and $\lim\limits_{x\to-\infty} F_X(x)=0$.
		\mitem $F_X$ is continuous from the right and its limit exists from the left at every real point.
	\end{msecenumerate}

\end{prop*}

\begin{proof}

	\begin{msecenumerate}[0pt]
		\mitem If $x<y$, then $\set{X\leq x}\subseteq\set{X\leq y}$, so
			\[ F_X(x) = \probof{X\leq x} \leq \probof{X\leq y} = F_X(y) \]
			As required.

		\mitem Let $a_n$ be any monotonic increasing sequence to $\infty$.
			Then the sets $\set{X\leq a_n}$ are also increasing, and therefore
			\[ \lim_{n\to\infty} \cumof[X]{a_n} = \lim_{n\to\infty}\probof{X\leq a_n} = \probof{\bigcup_{n\in\bN}\set{X\leq a_n}} \]
			And the union of all sets $\set{X\leq a_n}$ is $\Omega$, since for every $\omega\in\Omega$, at some point $X(\omega)\leq a_n$.
			So:
			\[ = \probof\Omega = 1 \]
			Since this is true for any monotonic increasing sequence to $\infty$, it must be true for any sequence whose limit is infinity,
			and therefore
			\[ \lim_{x\to\infty} \cumof[X]x = 1 \]

			Let $a_n$ be any monotonic decreasing sequence to $-\infty$.
			Then the sets $\set{X\leq a_n}$ are also decreasing and therefore:
			\[ \lim_{n\to\infty}\cumof[X]{a_n} = \lim_{n\to\infty}\probof{X\leq a_n} = \probof{\bigcap_{n\in\bN}\set{X\leq a_n}} \]
			And the intersection of all sets $\set{X\leq a_n}$ is the empty set, as for every $\omega\in\Omega$, at some point $X(\omega)>a_n$.
			So
			\[ = \probof\varnothing = 0 \]
			And for the same reason as above, this means
			\[ \lim_{x\to-\infty}\cumof[X]x = 0 \]

		\mitem Let $\epsilon_n$ be a positive monotonic decreasing sequence to $0$.
			Then:
			\[ \lim_{n\to\infty}\cumof[X]{x+\epsilon_n} = \lim_{n\to\infty}\probof{X\leq x+\epsilon_n} =
			\probof{\bigcap_{n\in\bN}\set{X\leq x+\epsilon_n}} = \probof{\bigcap_{n\in\bN}X^{-1}\big((-\infty,x+\epsilon_n)\big)} \]
			And this intersection is equal to $X^{-1}\big((-\infty,x)\big]$, so this is equal to:
			\[ = \probof{X^{-1}\big((-\infty,x)\big]} = \probof{X\leq x} = \cumof[X]x \]
			And therfore for every $x_n\searrow x$, the limit of $\cumof[X]{x_n}$ is $\cumof[X]x$, so
			\[ \lim_{t\to x^+}\cumof[X]t = \cumof[X]x \]

			And if $\epsilon_n$ is a negative increasing sequence to $0$ then
			\[ \lim_{n\to\infty}\cumof[X]{x+\epsilon_n} = \probof{\bigcup_{n\in\bN}\set{X\leq x+\epsilon_n}} = \probof{X<x} \]
			(Recall that by the definition of a limit, $\epsilon_n$ will never equal $0$.)
			So
			\[ \lim_{t\to x^-}\cumof[X]t = \probof{X<x} \]
			So the limit exists.
	\end{msecenumerate}

	\hfill$\qed$

\end{proof}

\newpage
\begin{defn*}

	A random variable $X$ is \ppemph{discrete} if and only if there exists a countable set $B\in\borelof\bR$ such that $\probof{X\in B}=1$.

	\begin{note}

		This is consistent with our previous definition, since singletons are in $\borelof\bR$, and $B$ is countable:
		\[ \probof{X\in B} = \sum_{x\in B}\probof{X=x} = 1 \]
		So we can define a mass probability function:
		\[ \mprob_X\colon\Omega\longrightarrow[0,1] \]
		By $\mprobof[X]x = \probof{X=x}$.
		Since for every $x\notin B$, $\probof{X=x}=0$ (otherwise the sum couldn't be $1$), this creates a probability distribution.

	\end{note}

	And a random variable $X$ is \ppemph{continuous} if for every $x\in\bR$, $\probof{X=x}=0$.
	Note that this means a random variable can't be both discrete and continuous.

\end{defn*}

\begin{prop*}

	A random variable is continuous if and only if its cumulative distribution function is continuous.

\end{prop*}

\begin{proof}

	On one hand, if $X$ is continuous, then we need to show that $F_X$ is left-continuous (since we already know it is right-continuous).
	We know that
	\[ \lim_{t\to x^-}\cumof[X]t = \probof{X<x} \]
	And we also know that $\probof{X\leq x} = \probof{X<x} + \probof{X=x} = \probof{X<x}$.
	Therefore $F_X$ is also left continuous, as required.

	For the converse, we know that for every $x$:
	\[ \lim_{t\to x^-}\cumof[X]t = \cumof[X]x = \probof{X\leq x} \]
	Since $F_X$ is continuous.
	But we know the limit is $\probof{X<x}$, so $\probof{X\leq x}=\probof{X<x}+\probof{X=x}=\probof{X<x}$, so $\probof{X=x}=0$.
	Since this is true for every $x\in\bR$, $X$ is continuous.

	\hfill$\qed$

\end{proof}

\begin{exam}

	It is possible for a random variable to be neither continuous nor discrete.
	Let $X$ be a random variable over the probability space $\tuple{[0,1],\borelof{[0,1]},\prob}$
	defined by $X(\omega)=\minof{\omega,\frac12}$.
	So $\probof{X=\frac12}=\probof{\bracks{\frac12,1}}=\frac12$, so $X$ is not continuous.
	But suppose $X$ is discrete, so there exists a countable $S$ such that $\probof{X\in S}=1$.
	It is obvious then that $\frac12\in S$.
	But for every other $x\neq\frac12$, $\probof{X=x}=0$.
	So we get that:
	\[ 1 = \sum_{x\in S}\probof{X=x} = \frac12+\sum_{\frac12\neq x\in S}\probof{X=x} = \frac12 \]
	So we have that $1=\frac12$ in contradiction.

\end{exam}

\begin{lemm*}

	If $X$ is a random variable and if $\set{x_n}_{n=1}^\infty$ is a sequence of distinct real points, then
	\[ \lim_{n\to\infty}\probof{X=x_n} = 0 \]

\end{lemm*}

\begin{proof}

	We know that:
	\[ \probof{X\in\bigcup_{n\in\bN}x_n} = \sum_{n=1}^\infty\probof{X=x_n} \leq 1\]
	And since $\probof{X=x_n}$ is nonnegative, this sum must converge.
	And since the sum converges, the limit of $\probof{X=x_n}$ must be $0$, as required.

	\hfill$\qed$

\end{proof}

This means that for every $a\in\bR$, the limit of $\probof{X=x}$ as $x$ approaches $a$ is $0$.
This is because for every strictly monotonic sequence of points $x_n$ to $a$, $\probof{X=x_n}$ has a limit of $0$.
So the limit of $\probof{X=x}$ is $0$.

Notice then that $X$ is continuous if and only if $\probof{X=x}$ is continuous.
This is true because if $X$ is continuous then the limit of $\probof{X=x}$ as $x$ approaches $a$ is $0$, which is the
same as $\probof{X=a}$.
And if $\probof{X=x}$ is continuous then for every $a$, the limit of $\probof{X=x}$ as $x$ approaches $a$ is
$\probof{X=a}$, and that limit is $0$, so $\probof{X=a}=0$.

\begin{prop*}

	\[ \probof{X\in(a,b)} = \lim_{x\to a^+}\probof{X\in(x,b)} = \lim_{x\to b^-}\probof{X\in(a,x)} \]

\end{prop*}

\begin{proof}

	We know that
	\[ \probof{X\in(x,b)} = \probof{X<b} - \probof{X\leq x} = \probof{X<b} - \cumof[X]x \]
	And since $\cum_X$ is right-conttinuous, as $x$ approaches $a$ from right, the limit of this is:
	\[ \lim_{x\to a^+}\probof{X\in(x,b)} = \probof{X<b} - \cumof[X]a = \probof{X<b} - \probof{X\leq a} = \probof{X\in(a,b)} \]
	As required.

	And for the other equality:
	\[ \probof{X\in(a,x)} = \probof{X\leq x} - \probof{X\leq a} - \probof{X=x} = \cumof[X]x - \probof{X\leq a} - \probof{X=x} \]
	As $x$ approaches $b$ from the left, the limit of $\cumof[X]x$ is $\probof{X<b}$, and by the lemma above the limit of
	$\probof{X=x}$ is $0$.
	So:
	\[ \lim_{x\to b^-} = \probof{X<b} - \probof{X\leq a} = \probof{X\in(a,b)} \]
	As required.

	\hfill$\qed$

\end{proof}

\begin{prop*}

	\[ \probof{X\in(-\infty,a)} = \lim_{c\to-\infty}\probof{X\in(c,a)} \]
	And
	\[ \probof{X\in(a,\infty)} = \lim_{c\to\infty}\probof{X\in(a,c)} \]

\end{prop*}

\begin{proof}

	Recall that
	\[ \probof{X\in(c,a)} = \probof{X<a} - \cumof[X]c \]
	And the limit of $\cumof[X]c$ as $c$ goes to $-\infty$ is $0$ as shown above, so the limit of this is $\probof{X<a}$, which is
	$\probof{X\in(-\infty,a)}$ as required.

	Notice that
	\[ \probof{X\in(a,c)} = 1 - \probof{X\leq a\text{ or }X\geq c} = 1 - \probof{X\leq a} - \probof{X\geq c} =
	\probof{X<c} - \probof{X\leq a} = \cumof[X]c - \probof{X\leq a} - \probof{X=c} \]
	The limit of $\cumof[X]c$ is $1$ as shown above, and the limit of $\probof{X=c}$ is $0$ as shown in the lemma above.
	So the limit of this is $1-\probof{X\leq a} = \probof{X>a}$ which is $\probof{X\in(a,\infty)}$ as required.

	\hfill$\qed$

\end{proof}

\begin{defn*}

	A random variable $X$ is \ppemph{absolutely continuous} if there exists a real nonnegative function $f_X$ such that for every real $a<b$:
	\[ \probof{X\in(a,b)} = \int_a^b f_X(x)\,dx \]
	$f_X$ is called $X$'s \ppemph{probability density function}.

\end{defn*}

\begin{prop*}

	If $X$ is an absolutely continuous random variable then
	\begin{msecenumerate}
		\mitem $X$ is continuous.
		\mitem \[ \int_{-\infty}^\infty f_X(x)\,dx = 1 \]
		\mitem \[ \cumof[X]s = \int_{-\infty}^s f_X(x)\,dx \text{ and } \ccumof[X]s = \int_s^\infty f_X(x)\,dx \]
		\mitem \[ f(x) = F'(x) \]
	\end{msecenumerate}

\end{prop*}

\begin{proof}

	\begin{msecenumerate}[0pt]
		\mitem Suppose $a\in\bR$.
			Then:
			\[ \probof{X=a} = \probof{\bigcap_{n\in\bN}\set{X\in(a,a+\frac1n)}} = \lim_{n\to\infty}\probof{X\in\parens{a,a+\frac1n}} =
			\lim_{n\to\infty}\int_a^{a+\frac1n}f_X(x)\,dx \]
			And this limit approaches $0$.
			So $\probof{X=a}=0$ as required.

		\mitem Let $x_n$ be a sequence of monotonically increasing values to infinity, and $x'_n$ decreasing to negative infinity.
			Then:
			\[ \int_{-\infty}^\infty f_X(x)\,dx = \lim \int_{x'_n}^0 f_X(x) + \int_0^{x_n} f_X(x) = \lim \probof{X\in(x'_n,0)} + \probof{X\in(x_n,0)} \]
			This is equal to
			\[ \lim\probof{X\in(x'_n,x_n)} \]
			Since $X$ is continuous, so we can add in $\probof{X=0}$ as it is equal to $0$.
			And this is an increasing series so this is equal to:
			\[ \probof{\bigcup_{n\in\bN}\set{X\in(x'_n,x_n)}} = \probof{X\in\bR} = 1 \]
			As required.

		\mitem We can use a similar proof as above for this.
		\mitem Suppose $\phi$ is an antiderivative of $f_X$ (which must exist since $f_X$ is integrable over every interval), then:
			\[ F_X(t) = \int_{-\infty}^t f_X(x)\,dx = \lim_{s\to-\infty}\phi(t)-\phi(s) = \phi(t) - c \]
			Where $c$ is some constant.
			And if we differentiate both sides we get $F_X'(t)=f_X(t)$, as required.
	\end{msecenumerate}

	\hfill$\qed$

\end{proof}

\begin{defn*}

	If if $I$ is an interval with length $\ell$ (so $I=[a,b]$ or $(a,b)$, etc. and $\ell=b-a$) and $X$ is a random variable with a distribution:
	\[ f_X(x) = \frac1\ell\cdot\bone_{I}(x) \]
	Then $X$ has a \ppemph{uniform distribution} over $I$, this is denoted $X\sim\puniof I$.

\end{defn*}

It is simple to verify that this is a valid probability density function, since its integral over $\bR$ is equal to:
\[ \int_a^b\frac1{b-a}\,dx = \frac{b-a}{b-a} = 1 \]

\begin{defn*}

	If $X$ is a random variable with a probability density function:
	\[ f_X(t) = \lambda\cdot e^{-\lambda t}\cdot\bone_{[0,\infty)} \]
	Where $\lambda>0$, then $X$ has a \ppemph{exponential probability distribution}, denoted $X\sim\expof\lambda$.

\end{defn*}

This is a probability density function since its integral over $\bR$ is:
\[ \int_0^\infty \lambda e^{-\lambda t}\,dt = e^{-\lambda t}\big|_0^\infty = 1 \]

