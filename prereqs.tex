\begin{defn*}

	In order to reduce any ambiguity, I will define the following substitute for the set of naturals:

	Given $m\in\bZ$:
	\[ \bN_m\coloneqq\set{n\in\bZ}[n\geq m] \]

\end{defn*}

\begin{defn*}

	Since it pops up a lot in combinatorics and probability, it is useful to define:
	\[ \bracks{n}\coloneqq\set{m\in\bN_1}[m\leq n]\]

\end{defn*}

\begin{defn*}

	I define $\powsetof[k]{A}$ to be the set of all $k$-length subsets of $A$:
	\[ \powsetof[k]{A}\coloneqq\set{B\subseteq A}[\abs{B}=k] \]

	Suppose the cardinality of $A$ is $n$. I'll define the cardinality of $\powsetof[k]{A}$ to be:
	\[ \binom{n}{k}\coloneqq\abs{\powsetof[k]{A}} \]
	This is called a \ppemph{binomial coefficient}.

\end{defn*}

\begin{defn*}

	The \ppemph{symmetric group} of a set $A$, denoted $S_A$, is the set of all bijections from $A$ to itself.

	$S_n$ is another and more common way of writing $S_{[n]}$.

\end{defn*}

