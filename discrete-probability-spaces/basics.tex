\begin{note}

Given the sum:
\[ \sum_{x\in X}f(x) \]
Where $X$ is potentially uncountable, we define it to be:
\[ \sum_{x\in X} f(x) \coloneqq \sum_{x\in\sprtof[f]{X}} f(x) \]
Where $\sprtof[f]{X}$ is the \ppemph{support} of $f$ on $X$, defined to be:
\[ \sprtof[f]{X}\coloneqq\set{x\in X}[f(x)\neq0] \]

Note that the sum must still be absolutely convergant, as we're summing in any order.

\end{note}

\begin{defn*}

A probability space $\tuple{\Omega, \cF, \prob}$ is \ppemph{discrete} if there exists a function:
\[ P\colon\Omega\longrightarrow[0,1] \]
Such that for every event $A$:
\[ \probof{A} = \sum_{\omega\in A}\mprobof\omega \]

\end{defn*}

\begin{defn*}

A (finite) discrete probability space is \ppemph{uniform} if for every $\omega\in\Omega$:
\[ \mprobof\omega=\frac1{\abs{\Omega}} \]

\end{defn*}

\begin{prop*}

If $\tuple{\Omega,\cF,\prob}$ is a uniform probability space, then:
\[ \probof{A} = \frac{\abs{A}}{\abs\Omega} \]

\end{prop*}

\begin{proof}

This is quite simple to prove.
Notice:
\[ \probof{A} = \sum_{\omega\in A}\mprobof\omega = \sum_{\omega\in A}\frac1{\abs\Omega} = \frac{\abs{A}}{\abs\Omega} \]
As required.

\hfill$\qed$

\end{proof}

\newpage
\begin{exam}

Suppose you live on the planet Ekato-Karpouzia, which has $k$ days in its year.

You're in a class with $n$ people, what is the probability that at least two of the people in the class have the same
birthday? (Assuming $n\leq k$.)

This is called the \ppemph{birthday problem} or the \ppemph{birthday paradox}, since the probability of this is higher
than one would expect.

\end{exam}

\begin{blankpp}

Here, we can define $\Omega$ to be the set of all functions from $[n]$ to $[k]$, where $f(i)$ gives the birthday of
the $i$th person.

We also know that the probability space must be uniform, since the probability of having one distribution of birthdays is
the same as having another (this isn't the case in real life, but this is a good enough assumption for us).

Notice that the event that there are at least two people in the class with the same birthday is the complement of the
event that no one in the class shares a birthday with anyone else.
This event is the set of all injective functions from $[n]$ to $[k]$.
We know there are $\frac{k!}{(k-n)!}$ of those, which means the probability that everyone has a different birthday is:
\[ \frac{k!}{(k-n)!\cdot k^n} \]

This is the probability of the complement of our goal event, which means the probability we're looking for is:
\[ 1 - \frac{k!}{(k-n)!\cdot k^n} \]

Now suppose that E.~Karpouzia has the same number of days in a year as us, $365$.
That is, $k=365$.
If this were the case, even with $50$ people, there'd be a probability greater than $0.97$.

\end{blankpp}

\begin{thrm*}[totProbOneTheorem,Law\ of\ Total\ Probability\ Version\ One]

If $\set{A_i}_{i\in I}\in\cF$ is a countable partition of $\Omega$, that is:
\[ \bigsqcup_{i\in I} A_i = \Omega \]
Then for every event $B$:
\[ \probof{B} = \sum_{i\in I}\probof{B\cap A_i} \]

\end{thrm*}

\begin{proof}

Notice that:
\[ \bigsqcup_{i\in I}B\cap A_i = B\cap\parens{\bigsqcup_{i\in I} A_i} = B\cap\Omega = B \]
Which means that:
\[ \probof{B} = \probof{\bigsqcup_{i\in I} B\cap A_i} \]
And since $I$ is countable, this is equal to:
\[ \implies \probof{B} = \sum_{i\in I}\probof{B\cap A_i} \]
As required.

\hfill$\qed$

\end{proof}

\begin{exam}

An $n$-sided die and a $k$-sided die are rolled.
What is the probability the result of the $n$-sided die is less than that of the $k$-sided die?

\end{exam}

\begin{blankpp}

In this case, $\Omega=[n]\times[k]$, where $(x,y)\in\Omega$ corresponds to a result of $x$ on the $n$-sided die and $y$
on the $k$-sided die.

The event we're trying to compute a probability for is:
\[ B = \set{(x,y)\in\Omega}[x<y] \]

Let's define for $a\in[k]$:
\[ A_a = [n]\times\set{a} = \set{(x,a)\in\Omega} \]

Thus:
\[ \probof{B} = \sum_{a=1}^k \probof{B\cap A_a} \]

$B\cap A_a=\set{(x,a)}[x<a]$, which means that $\abs{B\cap A_a}=a-1$ (as there are $a-1$ choices for $x$).
Since the probability is uniform (the probability of rolling any two numbers is equal), this means:
\[ \probof{B\cap A_a} = \frac{\abs{B\cap A_a}}{\abs\Omega} = \frac{a-1}{n\cdot k} \]

Therefore:
\[ \probof{B} = \sum_{a=1}^k \frac{a-1}{n\cdot k} = \frac1{nk}\cdot\sum_{a=0}^{k-1}a =
   \frac{1}{nk}\cdot\frac k2\cdot(k-1) = \frac{k-1}{2n} \]

\end{blankpp}

\begin{lemm*}[finiteUnionBoundLemma]

	If $\set{A_i}_{i=1}^n$ are events, then:
	\[ \probof{\bigcup_{i=1}^n A_i}\leq\sum_{i=1}^n\probof{A_i} \]

\end{lemm*}

\begin{proof}

	We can prove this through induction on $n=1$. The base case for $n=1$ is trivial.
	We will also prove it for $n=2$.

	\ppemph{Base case} $n=2$:

	Notice that:
	\[ \probof{A_1\cup A_2} = \probof{A_1\sqcup(A_2\setminus A_1)} = \probof{A_1} + \probof{A_2\setminus A_1} \]
	And since $A_2\setminus A_1$ is a subset of $A_2$, its probability is less than the probability of $A_2$.
	So we get:
	\[ \probof{A_1\cup A_2} = \probof{A_1} + \probof{A_2\setminus A_1} \leq \probof{A_1} + \probof{A_2} \]
	As required.

	\ppemph{Inductive step}:

	Suppose the hypothesis is true for $n$.
	Let $\set{A_i}_{i=1}^{n+1}$ be events.
	Then:
	\[ \probof{\bigcup_{i=1}^{n+1} A_i} = \probof{\parens{\bigcup_{i=1}^n A_i}\cup A_{n+1}} \]
	And as we proved for $n=2$, this is less than:
	\[ \leq \probof{\bigcup_{i=1}^n A_i} + \probof{A_{n+1}} \]
	And by our inductive hypothesis, $\probof{\bigcup\limits_{i=1}^n A_i}\leq\sum\limits_{i=1}^n\probof{A_i}$, so this is 
	less than:
	\[ \leq \sum_{i=1}^n\probof{A_i} + \probof{A_{n+1}} = \sum_{i=1}^{n+1}\probof{A_i} \]

	As required.

\hfill$\qed$

\end{proof}

\begin{defn*}

	A series of sets $\set{A_i}_{i=1}^\infty$ is \ppemph{increasing} if for every $n\in\bN_1$, $A_n$ is a subset of $A_{n+1}$.
	And it is \ppemph{decreasing} if for every $n\in\bN_1$, $A_n$ is a superset of $A_{n+1}$.

\end{defn*}

\begin{thrm*}[contProbTheorem]

	If $\set{A_i}_{i=1}^\infty$ is increasing, then:
	\[ \probof{\bigcup_{n=1}^\infty A_n} = \lim_{n\to\infty}\probof{A_n} \]

\end{thrm*}

\begin{proof}

	Let's define $B_n\coloneqq A_n\setminus A_{n-1}$.

	\begin{claim}
		$\set{B_n}_{n=1}^\infty$ is pairwise disjoint.

		Suppose $\omega\in B_n$, then $\omega\notin A_{n-1}$.
		Since $\set{A_i}_{i=1}^\infty$ is increasing, this means that for every $k<n$, we know $\omega\notin A_k$.
		And since $B_k\subseteq A_k$, this means that for every $k<n$, $\omega\notin B_k$.

		Suppose that for some $k>n$, $\omega\in B_k$. But this would mean $\omega\notin B_n$ (by above), which is a contradiction.

		So for every $k\neq n: \omega\notin B_k$.
		So $\set{B_n}_{n=1}^\infty$ is disjoint.
	\end{claim}

	\begin{claim}
		\[ \bigsqcup_{k=1}^n B_k = A_n \]

		Suppose $\omega\in\bigsqcup B_k$, then there exists some $k\leq n$ such that $\omega\in B_k$.
		And since $B_k\subseteq A_k$, this means $\omega\in A_k$.

		Now suppose $\omega A_n$.
		There must be some $k$ such that $\omega\in A_k$ and $\omega\notin A_{k-1}$
		(take $k=\min\set{m\leq n}[\omega\in A_m]$).
		This means $\omega\in A_k\setminus A_{k-1}=B_k$, and $k\leq n$, so $\omega\in\bigsqcup B_k$.

		So an element is in one union if and only if it is in $A_n$, and therefore they are equal.
	\end{claim}

	It follows that:
	\[ \bigsqcup_{n=1}^\infty B_n = \bigcup_{n=1}^\infty A_n \]
	Since:
	\[ \bigcup_{n=1}^\infty A_n = \bigcup_{n=1}^\infty \bigsqcup_{k=1}^n B_k = \bigsqcup_{n=1}^\infty B_k \]

	So:
	\[ \probof{\bigcup_{n=1}^\infty A_n} = \probof{\bigsqcup_{n=1}^\infty B_n} = \sum_{n=1}^\infty \probof{B_n} \]
	Now recall that by definition:
	\[ \sum_{k=1}^\infty\probof{B_k} = \lim_{n\to\infty}\sum_{k=1}^n\probof{B_k} = \lim_{n\to\infty}\probof{\bigsqcup_{k=1}^n B_k} \]
	And by our previous claim, this is equal to:
	\[ = \lim_{n\to\infty}\probof{A_n} \]
	As required.

\hfill$\qed$

\end{proof}

\begin{coro*}[contProbCoro]

	If $\set{A_i}_{i=1}^\infty$ are events which are decreasing, then:
	\[ \probof{\bigcap_{n=1}^\infty A_n} = \lim_{n\to\infty}\probof{A_n} \]

\end{coro*}

\begin{proof}

	Notice that since $A_n\supseteq A_{n+1}$, we know that $A_n^c\subseteq A_{n+1}^c$, so $\set{A_i^c}_{i=1}^\infty$ is an
	increasing series.
	Furthermore:
	\[ \bigcap_{n=1}^\infty A_n = \parens{\bigcup_{n=1}^\infty A_n^c}^c \]
	So:
	\[ \probof{\bigcap_{n=1}^\infty A_n} = 1 - \probof{\bigcup_{n=1}^\infty A_n^c} \]

	And since $\set{A_i^c}_{i=1}^\infty$ is increasing, by the above theorem:
	\[ = 1 - \lim_{n\to\infty}\probof{A_n^c} = \lim_{n\to\infty} 1-\probof{A_n^c} = \lim_{n\to\infty}\probof{A_n} \]
	As required.

\hfill$\qed$

\end{proof}

\begin{note}

	These two theorems will become very important in the future when we discuss general probability spaces.
	So keep them in mind.

\end{note}

\newpage
\begin{thrm*}

	If $I$ is countable then:
	\[ \probof{\bigcup_{i\in I} A_i}\leq\sum_{i\in I}\probof{A_i} \]

\end{thrm*}

\begin{proof}

	We already proved the finite case in \ppref[lemma]{finiteUnionBoundLemma},
	so all that remains is to prove the countably finite case.
	That is, we need to prove:
	\[ \probof{\bigcup_{n=1}^\infty A_n} \leq \sum_{n=1}^\infty\probof{A_n} \]

	Let:
	\[ B_n\coloneqq\bigcup_{k=1}^n A_k \]
	Then $\set{B_n}_{n=1}^\infty$ is increasing, and:
	\[ \bigcup_{n=1}^\infty B_n = \bigcup_{n=1}^\infty A_n \]

	So by the previous theorem:
	\[ \probof{\bigcup_{n=1}^\infty A_n} = \probof{\bigcup_{n=1}^\infty B_n} = \lim_{n\to\infty}\probof{B_n} =
	\lim_{n\to\infty}\probof{\bigcup_{k=1}^n A_k} \]

	And by \ppref[lemma]{finiteUnionBoundLemma}:
	\[ \probof{\bigcup_{k=1}^n A_k}\leq\sum_{k=1}^n\probof{A_k} \]
	So:
	\[ \lim_{n\to\infty}\probof{\bigcup_{k=1}^n A_k}\leq\lim_{n\to\infty}\sum_{k=1}^n\probof{A_k} = \sum_{n=1}^\infty\probof{A_n} \]

	Which means that all in all:
	\[ \probof{\bigcup_{n=1}^\infty A_n} \leq \sum_{n=1}^\infty\probof{A_n} \]
	As required.

\hfill$\qed$

\end{proof}

