\begin{prop*}

	There are $\frac{n!}{(n-k)!}$ injective functions from $[k]$ to $[n]$.

\end{prop*}

\begin{proof}

	Let $A$ be this set of injective functions.

	Let's create a bijection:
	\[ f\colon[n]\times[n-1]\times\cdots\times[n-k+1]\longrightarrow A \]
	We know that forall $a_1,\dots,a_k\in[n]$ there exists a bijection:
	\[ f_{\set{a_1,\dots,a_k}}\colon [n-k]\longrightarrow[n]\setminus\set{a_1,\dots,a_k} \]
	So we can define $f$ like so:
	\[ f(a_1,a_2,\dots,a_{n-k}) = g \]

	Where:
	\[ g(1) = a_1 \]
	And:
	\[ g(i) = f_{\set{g(1),\dots,g(i-1)}}(a_i) \]
	For $i\geq2$.

	This is well-defined and $g$ is an injection since $g(i)\in[n]\setminus\set{g(1),\dots,g(i-1)}$ (this can be shown
	inductively. It is not trivial since we must prove that $g(1)\neq\cdots g(i-1)$ in order to show that
	$f_{\set{g(1),\dots,g(i-1)}}$ is the intended bijection.)

	$f$ is injective since if:
	\[ f(a_1,\dots,a_{n-k}) g_1 = g_2 = f(b_1,\dots,b_{n-k}) \]
	Then:
	\[ g_1(1) = g_2(1) \implies a_1 = b_1 \]
	And:
	\[ g_1(2) = g_2(2) \implies f_{\set{a_1}}(a_2) = f_{\set{a_1}}(b_2) \implies a_2=b_2 \]
	Since $f_{\set{a_1}}$ is a bijection. 

	And so on.

	Now we must prove that $f$ is surjective. Suppose $g$ is an injection. Notice then that:
	\[ f(g(1), f^{-1}_{\set{g(1)}}(g(2)), f^{-1}_{\set{g(1),g(2)}}(g(3)), \dots, f^{-1}_{\set{g(1),\dots,g(k-1)}}(g(k))) = g \]
	Let $h$ be the left side ($f(\dots)$), then:
	\[ h(i) = f_{\set{g(1),\dots,g(i-1)}}\parens{f^{-1}_{\set{g(1),\dots,g(i-1)}}(g(i))} = g(i) \]
	So $h=g$ as required.
	
	Therefore $f$ is a bijection and:
	\[ \abs{A} = n\cdot(n-1)\cdots(n-k+1) = \frac{n!}{(n-k)!} \]
	As required.

	\hfill$\qed$

\end{proof}

\begin{note}

	This proof is just a more rigorous wording of the classic construction of the injective functions.

	To construct an injective function, first choose the image of $1$. There are $n$ choices for this. Then there are $n-1$
	choices for $2$, and so on. In the end we get that in total there are:
	\[ n\cdot(n-1)\cdots(n-k+1) = \frac{n!}{(n-k)!} \]
	choices and thus injective functions.

\end{note}

\def\image{{\rm Im}}
\begin{lemm*}

	If $A$ and $B$ are finite sets with the same cardinality, then every injection from $A$ to $B$ is a bijection.

\end{lemm*}

\begin{proof}

	Suppose $f\colon A\longrightarrow B$ is an injection.
	This means that $\abs{\image f}=\abs{A}=\abs{B}$. 

	Suppose, for the sake of a contradiction, that $f$ is not a surjection.
	Then there exists a $b\in B$ which has not origin in $A$, that is $b\notin\image f$.

	But we know that $\image f\sqcup\set{b}\subseteq B$, so:
	\begin{align*}
		\abs{\image f\sqcup\set{b}} &\leq \abs{B} \\
		\implies \abs{\image f}+1 &\leq \abs{B} \\
		\implies \abs{B}+1 &\leq \abs{B} \\
		\implies 1\leq 0
	\end{align*}

	In contradiction.
	So $f$ is a surjection, and therefore a bijection, as required.

\hfill$\qed$

\end{proof}

\begin{thrm*}

	There are $n!$ permutations of a set of cardinality $n$. That is, $\abs{S_n}=n!$.

\end{thrm*}

\begin{proof}

	We know that every injection from $[n]$ to $[n]$ is a bijection, and vice versa.
	So $S_n$ is the set of injections from $[n]$ to $[n]$, of which we proved there are:
	\[ \frac{n!}{(n-n)!} = n! \]
	As required.

\hfill$\qed$

\end{proof}

\newpage
\begin{lemm*}[partCardLemma]

	If $P$ is a partition of $A$ such that every equivalence class has equal cardinality:
	\[ \abs{A} = \abs{P}\cdot p \]
	Where $p$ is the cardinality of an equivalence class.

\end{lemm*}

\begin{proof}

	Suppose $[a]$ is an equivalence class of $A$. Then we know that for every equivalence class $[b]$, there exists a
	bijection:
	\[ f_{[b]}\colon [a]\longrightarrow[b] \]
	Since they have equal cardinalities.

	We'll define a function:
	\[ f\colon P\times[a]\longrightarrow A \]
	Where:
	\[ f([b], \alpha) = f_{[b]}(\alpha) \]

	This is injective since:
	\[ f([b], \alpha) = f([c], \beta) \implies f_{[b]}(\alpha) = f_{[c]}(\beta) \]
	Since the codomain of $f_{[x]}$ is $[x]$, and if $[x]\neq[y]$ then $[x]\cap[y]=\varnothing$ as $P$ is a partition, this
	means that $[b]=[c]$. And since $f_{[b]}$ is a bijection, this means that $\alpha=\beta$.

	So $([b],\alpha)=([c],\beta)$, which means $f$ is injective.
	
	Now, suppose $b\in A$, then:
	\[ f([b], f_{[b]}^{-1}(b)) = f_{[b]}\parens{f_{[b]}^{-1}(b)} = b \]
	So $f$ is surjective.

	This means that $f$ is a bijection.

\hfill$\qed$

\end{proof}

\begin{prop}

	There are $(n-1)!$ distinct ways to place $n$ people around a circular table.

\end{prop}

\begin{proof}

	This is the same as asking how many distinct permutations there are if we define the equivalence class of a permutation
	$\sigma\in S_n$ by:
	\[ [\sigma]=\set{\tau\in S_n}[\exists i\in\bZ: \tau(x)=\sigma(x+i\bmod n)] \]
	As $x+i\bmod n$ corresponds to a shift of $i$ spots about the table.

	We know for all $i\in\bZ$, $x+i\bmod n=x+(i\bmod n)\bmod n$, and $0\leq i\bmod n<n$. So:
	\[ [\sigma]=\set{\tau\in S_n}[\exists 0\leq i<n: \tau(x)=\sigma(x+i\bmod n)] \]
	And for every $0\leq i\neq j<n$:
	\[ x+i\bmod n\neq x+j\bmod n\implies \sigma(x+i\bmod n)\neq\sigma(x+j\bmod n) \]
	Therefore:
	\[ \abs{[\sigma]} = n \]

	By \ppref[lemma]{partCardLemma}, this means that the number of distinct permutation is:
	\[ \frac{n!}{n} = (n-1)! \]
	As required.

\hfill$\qed$

\end{proof}

\begin{thrm*}[twoObjOrderingsTheorem]

	The number of distinct ways to order $k$ black balls and $n-k$ white ones is:
	\[ \frac{n!}{k!\cdot(n-k)!} \]

\end{thrm*}

\begin{proof}

	Let the balls form the sequence $\set{a_i}_{i=1}^n$ where $a_i$ is $1$ (black) for $i\leq k$ and $0$ (white) for all other
	$a_i$s. We define the equivalence class between permutations of the $n$ balls:
	\[ [\sigma] = \set{\tau\in S_n}[a_{\tau(i)} = a_{\sigma(i)}] \]
	This represents all the permutations of the balls which give the same ordering as $\sigma$. We want to count the number
	of distinct permutations there are, which is the number of distinct equivalence classes.

	Let:
	\[ A_\sigma\coloneqq\set{n\geq i\in\bN_1}[a_{\sigma(i)}=1] \]
	And:
	\[ B_\sigma\coloneqq[n]\setminus A_\sigma \]
	We will define a bijection:
	\[ f\colon S_{A_\sigma}\times S_{B_\sigma}\longrightarrow[\sigma] \]

	Where:
	\[ f(\sigma_A, \sigma_B) = \tau \]
	Where $\tau$ is defined by:
	\[ \tau(x) = \begin{cases} \sigma_A(x) & x\in A_\sigma \\ \sigma_B(x) & x\in B_\sigma \end{cases} \]

	We need to show that this is well-defined. Firstly, this is a bijection because $A_\sigma$ and $B_\sigma$ are disjoint and
	$\sigma_A$ and $\sigma_B$ are bijections.
	Suppose $i\in[n]$, if $i\in A_\sigma$ then by definition $a_{\sigma(i)}=1$, and $a_{\tau(i)} = a_{\sigma_A(i)}$,
	but $\sigma_A(i)\in A_\sigma$, so $a_{\sigma_A(i)}=1$, as required.
	The proof is similar if $i\in B_\sigma$.

	Now, let's show that $f$ is injective. Suppose:
	\[ f(\sigma_A, \sigma_B) = \tau_1 = \tau_2 = f(\pi_A, \pi_B) \]
	This means that for every $i\in A_\sigma$: $\tau_1(i)=\tau_2(i)$, but we know that this just means $\sigma_A(i)=\pi_A(i)$,
	therefore $\sigma_A=\pi_A$. Similar for $B$. Since the two sets are finite, this means that $f$ is a bijection.
	
	And we know that the cardinality of $A_\sigma$ is $k$ (since $\sigma$ is a bijection and there are $k$ $a_i$s which equal
	$1$), and therefore $B_\sigma$ has a cardinality of $n-k$. So:

	\[ \abs{\sigma} = k!\cdot(n-k)! \]
	
	So by \ppref[lemma]{partCardLemma}, this means the number of equivalence classes is:
	\[ \frac{n!}{k!\cdot(n-k)!} \]

	\hfill$\qed$

\end{proof}

\begin{coro*}[binomialIntro]

	\[ \binom{n}{k} = \frac{n!}{k!\cdot(n-k)!} \]

\end{coro*}

\begin{proof}

	Let $\set{a_i}_{i=1}^n$ be defined similarly to as it was above. Let $P$ the set of distinct orderings of this set.
	As per the theorem above, $\abs{P}=\frac{n!}{k!\cdot(n-k)!}$. We will construct a bijection:
	\[ f\colon P\longrightarrow\powsetof[k]{[n]} \]

	We define it like so:
	\[ f([\sigma]) = \set{m\in[n]}[a_{\sigma(m)}=1] \]

	We know this must have a cardinality of $k$ since there are $k$ $a_i$s which are equal to $1$, and permutations in the
	same equivalence class map the same $a_i$s to $1$, so the function is well-defined.

	Let us prove that $f$ is injective. If $f(\sigma)=f(\tau)$ then
	\[ a_{\sigma(m)} = 1 \iff a_{\tau(m)} = 1 \]
	Which means that $a_{\sigma(m)}=a_{\tau(m)}$, and therefore $[\tau]=[\sigma]$, as required.

	Now suppose $S=\set{x_1,\dots,x_k}\in\powsetof[k]{[n]}$. We can define a permutation like so:
	\[ \sigma(x_i) = i \]
	And since $\abs{[n]\setminus S}=n-k$, there is a bijection between $[n]\setminus S$ and $\set{k+1,\dots,n}$. So we can
	map the values in $[n]\setminus S$ (non-$x_i$ values) bijectively to $\set{k+1,\dots,n}$.

	This defines a bijection $\sigma$ in $S_n$. Now notice that since $a_i=1$ only for $i\leq k$:
	\[ \set{m\in[n]}[a_{\sigma(m)}=1] = \set{m\in[n]}[\sigma(m)\leq k] = \set{x_1,\dots,x_k} = S \]
	So:
	\[ f([\sigma]) = S \]
	And $f$ is therefore a surjection.

	Therefore $f$ is bijective and:
	\[ \binom{n}{k} = \abs{\powsetof[k]{[n]}} = \abs{P} = \frac{n!}{k!\cdot(n-k)!} \]
	As required.

	\hfill$\qed$

\end{proof}

\newpage
\begin{prop}[pascalsIdentity,Pascal's\ Identity]

	\[ \binom{n}{k} = \binom{n}{n-k} \]

	This identity is named after \ppemph{Blaise Pascal}, one of the pioneers of early probability theory.

\end{prop}

\begin{proof}

	\begin{note}

		This can be proved algebraically, by applying the formula for binomial coefficient which we proved earlier.

		But this proof is dry and does not reveal much about the inner nature of the identity.

	\end{note}

	This is a simple proof. All it requires is a construction of a bijection:
	\[ f\colon\powsetof[k]{[n]}\longrightarrow\powsetof[n-k]{[n]} \]
	The construction is quite simple and natural:
	\[ f(S) = [n]\setminus S \]
	This is obviously well-defined, and is a bijection since it is its own inverse.

\hfill$\qed$

\end{proof}

\begin{prop*}[pascalsRule,Pascal's\ Rule]

	\[ \binom{n-1}{k} + \binom{n-1}{k-1} = \binom{n}{k} \]

\end{prop*}

\begin{proof}

	A good look at the proposition reveals what the theorem really means: there is a partition of $\powsetof[k]{[n]}$ into
	two sets isomorphic to $\powsetof[k]{[n-1]}$ and $\powsetof[k-1]{[n-1]}$ respectively. So let's attempt to find (one) of
	these partitions.

	We can define the partition into two sets:
	\[ A=\set{S\in\powsetof[k]{[n]}}[n\in S] \]
	And:
	\[ B=\powsetof[k]{[n]}\setminus A = \set{S\in\powsetof[k]{[n]}}[n\notin S] \]

	First, let's show that $A$ is isomorphic to $\powsetof[k-1]{[n-1]}$. Let $S\in A$. We can map it to $S\setminus\set{n}$.
	This is well-defined since it has a cardinality of $\abs{S}-1=k-1$ and is a subset of $[n]\setminus\set{n}=[n-1]$.
	Since for every $S\in A$, $n\in S$, this is injective (since $S=S\setminus\set{n}\cup\set{n}$). 
	It is surjective since given $S'\in\powsetof[k-1]{[n-1]}$, $S'\cup\set{n}$ will be mapped to it.

	And $B$ is actually just equal to $\powsetof[k]{[n-1]}$. This is because if $S\in B$ then
	$S\subseteq[n]\setminus\set{n}=[n-1]$, and $S$ has a cardinality of $k$.

	And $A\sqcup B=\powsetof[k]{[n]}$ trivially.

	Therefore:
	\[ \abs{A} + \abs{B} = \abs{\powsetof[k]{[n]}} \implies \binom{n-1}{k-1}+\binom{n-1}{k} = \binom{n}{k} \]
	As required.

\hfill$\qed$

\end{proof}

\newfunc{mset}{{\cal M}}({})
\newfunc{sprt}{{\rm supp}}({})
\def\bU{\mathbb{U}}
\def\msetcoeff#1#2{\left(\!\binom{#1}{#2}\!\right)}

\begin{defn*}

	A \ppemph{multiset} in a universe $\bU$ is a pair:
	\[ A=(\bU, m) \]
	Where $\bU$ is a set and $m$ is a function:
	\[ m\colon \bU\longrightarrow\bN_0 \]
	This is called $A$'s \ppemph{multiplicity function} and is also denoted $m_A$.
	
	Multisets are a way of representing sets where elements can be repeated a finite number of times.
	Basically, for every $a$ in $A$, $m(a)$ represents how many times $a$ is in the multiset.

	Let $\msetof{\bU}$ be the set of all multisets in the universe $\bU$, that is:
	\[ \msetof{\bU}\coloneqq\set{(\bU, m)}[m\colon \bU\longrightarrow\bN_0] \]

	Given a multiset $A$ in a universe $\bU$, we define its \ppemph{support} to be the set of all elements in $A$:
	\[ \sprtof{A}\coloneqq\set{a\in\bU}[m(a)>0] \]

	And we say that $a\in A$ if $a\in\sprtof{A}$.

	We define the \ppemph{cardinality} of a finite multiset $A$ (a multiset is finite if its support is):
	\[ \abs{A}\coloneqq\sum_{a\in\sprt{A}}m(a) \]

	Finally, we define the set of all $k$-length multisets over $\bU$ as:
	\[ \msetof[k]{\bU} = \set{A\in\msetof{\bU}}[\abs{A}=k] \]
	And we define its cardinality to be:
	\[ \msetcoeff{\abs{\bU}}{k}\coloneqq\abs{\msetof[k]{\bU}} \]
	This is called the \ppemph{multiset coefficient}.

\end{defn*}

\begin{prop*}

	\[ \msetcoeff{n}{k} = \binom{n+k-1}{k} \]

\end{prop*}

\begin{proof}

	So we need to find the cardinality of $\msetof[k]{[n]}$.

	Firstly, recall that by \ppref[corollary]{binomialIntro}
	the number of distinct orderings of $k$ of one object and $n-1$ of another is $\binom{n+k-1}{k}$.
	So let's create a bijection from the set of distinct orderings of $k$ one object and $n-1$ of another.

	These orderings can be uniquely characterized by the indexes of the $n-1$ objects, which are the series
	$\set{p_i}_{i=1}^{n-1}$ where $1\leq p_1<\cdots<p_{n-1}\leq n+k-1$.
	For simplicity, let's define $p_0\coloneqq0$ and $p_n\coloneqq n+k$.

	\newfunc{func}{f}({})
	So now let's define the bijection:
	\[ \funcof{\set{p_i}_{i=0}^n} = ([n], m) \]
	Where for every $k\in[n]$:
	\[ m(k) = p_k - p_{k-1} - 1 \]

	Let's prove that this is well-defined.
	Firstly, $p_k>p_{k-1}\implies p_k-p_{k-1}-1\geq0$, so $m(k)\in\bN_0$ as required.

	Secondly:
	\[ \sum_{k=1}^n m(k) = \sum_{k=1}^n p_k - p_{k-1} - 1 = \sum_{k=1}^n\parens{p_k-p_{k-1}} - n \]
	This is a telescopic sum, which is equal to:
	\[ = p_n - p_0 - n = n+k - n = k \]
	Which means this multiset has a cardinality of $k$, as required.

	Now, let's show that $f$ is an injection.

	Suppose:
	\[ \funcof{\set{p_i}_{i=0}^n} = \funcof{\set{q_i}_{i=0}^n} \]
	That means for every $k\in[n]$:
	\[ p_k - p_{k-1} - 1 = q_k - q_{k-1} - 1 \implies p_k - p_{k-1} = q_k - q_{k-1} \]
	And through simple induction it can be shown that $p_k=q_k$ (recall that $p_0=q_0=0$ for the base).

	So $f$ is injective.

	Now, suppose $([n], m)\in\msetof[k]{[n]}$ is a multiset, we can define:
	\[ p_k\coloneqq\sum_{i=1}^k (m(i)) + k \]
	Which is a valid indexing since $p_k<p_{k+1}$, $p_0=0$, and $p_n=\sum\limits_{i=1}^k m(i) + n = k+n$.

	And notice that:
	\[ p_k - p_{k-1} = m(k) + 1 \implies m(k) = p_k - p_{k-1} - 1 \]
	Which means that:
	\[ \funcof{\set{p_i}_{i=0}^n} = ([n], m) \]
	So every multiset has an origin, therefore $f$ is surjective.

	So $f$ is a bijection, which means that:
	\[ \msetcoeff{n}{k} = \abs{\msetof[k]{[n]}} = \binom{n+k-1}{k} \]

	As required.

\hfill$\qed$

\end{proof}

\begin{prop*}

	\[ \msetcoeff{n}{k} = \msetcoeff{k+1}{n-1} \]

\end{prop*}

\begin{proof}

	As shown earlier, $\msetof[k]{[n]}$ is isomorphic to the set of orderings of $n-1$ of one object and $k$ of another.
	So for instance, it is isomorphic to the set of orderings of $n-1$ black balls and $k$ white balls.

	But we can flip which object is which, so this is isomorphic to the orderings of $k$ black balls and $n-1$ white balls.
	And this is isomorphic by the (proof of the) previous proposition to $\msetof[n-1]{[k+1]}$.

	So all in all $\msetof[k]{[n]}$ is isomorphic to $\msetof[n-1]{[k+1]}$, so:
	\[ \msetcoeff{n}{k} = \msetcoeff{k+1}{n-1} \]
	As required.

\hfill$\qed$

\end{proof}

\begin{prop*}

	\[ \msetcoeff{n}{k} = \msetcoeff{n}{k-1} + \msetcoeff{n-1}{k} \]

\end{prop*}

\begin{proof}

	We will create a partition of $\msetof[k]{[n]}$ into subsets which have the required cardinality.
	The partition which accomplishes this is defined as follows:
	\[ \msetof[k]{[n]} = A\sqcup B \]
	Where:
	\[ A\coloneqq\set{M\in\msetof[k]{[n]}}[n\notin M] \quad B\coloneqq\set{M\in\msetof[k]{[n]}}[n\in M] \]

	Notice that $A=\msetof[k]{[n-1]}$.

	We can define a bijection $f$ from $B$ to $\msetof[k-1]{[n]}$ as follows:
	\[ f(M) = \tilde M \]
	Where:
	\[ m_{\tilde M}(i) = \begin{cases}
		m_M(i) & i\neq n \\
		m_M(i) - 1 & i=n 
	\end{cases} \]

	This is well-defined because the sum of $m_{\tilde M}$ is one less than the sum of $m_M$, which is $k$.
	So the sum is $k-1$, as required.

	It is obviously injective, since we have all the necessary knowledge about $m$ in its image.

	It is also obviously surjective, since given $m_{\tilde M}$, we can define $m$ like so:
	\[ m(i) = \begin{cases} m_{\tilde M}(i) & i\neq n \\ m_{\tilde M}(i) + q & i=n \end{cases} \]

	So $f$ is a bijection, therefore:
	\[ \abs{B} = \msetcoeff{n}{k-1} \]

	So all in all we know:
	\[ \abs{\msetof[k]{[n]}} = \abs{A} + \abs{B} \implies \msetcoeff{n}{k} = \msetcoeff{n-1}{k} + \msetcoeff{n}{k-1} \]
	As required.

\hfill$\qed$

\end{proof}

\newpage
\begin{thrm*}

	Given a set $A$ with $n$ elements, the number of ways to choosen $k$ elements from $A$ is:
	\begin{msecitemize}
		\mitem If order matters (so choosing $a$ then $b$ is different from choosing $b$ then $a$)
		and repition is allowed (we can choose $a$ twice for example):
			\[ n^k \]
		\mitem If order matters, but repetition is not allowed:
			\[ \frac{n!}{(n-k)!} \]
		\mitem If order doesn't matter, but repetition is allowed:
			\[ \msetcoeff{n}{k} = \binom{n+k-1}{k} \]
		\mitem If order doesn't matter, and repetition is not allowed:
			\[ \binom nk \]
	\end{msecitemize}

	This should give insight as to why what we've been discussing is significant.

\end{thrm*}

\begin{proof}

	\begin{msecitemize}
		\mitem If order matters and repetition is allowed, then this is just analogous to the number of tuples over $A$ with
		length $k$, this is the set $A^k$, which has a cardinality of $n^k$.

		\mitem If order matters, but repetition is not allowed, then this is just analogous to the number of tuples over $A$
		with length $k$, but all elements in the tuples are distinct.
		And tuples are analogous to functions, so this is analogous to the injective functions from $[k]$ to $A$,
		of which there are $\frac{n!}{(n-k)!}$.

		\mitem If order doesn't matter, and repetition is allowed, then we're just choosing a multiset of length $k$ from $A$.
		And we know there are $\msetcoeff{n}{k}=\binom{n+k-1}{k}$ of those.

		\mitem If order doesn't matter and repetition is not allowed, then we're just choosing a set of cardinality $k$ from
		$A$, in other words a $k$-length subset of $A$.
		And we know there are $\binom{n}{k}$ of those.
	\end{msecitemize}

\hfill$\qed$

\end{proof}


