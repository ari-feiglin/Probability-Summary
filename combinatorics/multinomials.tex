\begin{defn*}

	Given a tuple $I$, $\matindex{I}_i$ is the element in the $i$-th position of the tuple.

	If $I$ has a size of $n$, then for a $\sigma\in S_n$, I define $\sigma(I)$ as the tuple where:
	\[ \matindex{\sigma(I)}_i = \matindex{I}_{\sigma(i)} \]

\end{defn*}

\begin{lemm*}

	\[ \matindex{(\sigma\circ\tau)(I)}_i = \matindex{\sigma(I)}_{\tau(i)} \]

\end{lemm*}

\begin{proof}

	Notice that:
	\[ \matindex{\sigma(I)}_{\tau(i)} = \matindex{I}_{\sigma\circ\tau(i)} \]
	And:
	\[ \matindex{(\sigma\circ\tau)(I)} = \matindex{I}_{\sigma\circ\tau(i)} \]
	As required.

\hfill$\qed$

\end{proof}

\begin{lemm*}[multinomialLemma]

	Suppose $S=\set{s_1,\dots,s_\ell}$.

	The number of tuples of length $n$ over $S$ which have $k_i$ occurrences of $s_i$ is:
	\[ \frac{n!}{k_1!\cdots k_\ell!} \]

	(Assuming $\sum\limits_{i=1}^\ell k_i = n$)

\end{lemm*}

\begin{proof}

	Let:
	\[ I = \tuple{\underbrace{s_1,\dots,s_1}_{k_1 \text{ times}},\dots,\underbrace{s_\ell,\dots,s_\ell}_{k_\ell \text{ times}}} \]
	All other tuples which satisfy the criteria are permutations of this one.

	We define equivalence the equivalence class of $\sigma\in S_n$ as the set of all permutations which produce the same
	tuple:
	\[ [\sigma] = \set{\tau\in S_n}[\sigma(I) = \tau(I)] \]
	We need to find the number of distinct equivalence classes there are. To do so we will find the cardinality of each
	equivalence class.

	Let:
	\[ \sigma_j = \set{i\in[n]}[\matindex{\sigma(I)}_i=s_j] \]
	For $j\in[\ell]$. This is the set of all indexes in the permuted tuple of the elements $s_j$
	(notice that $[\tau]=[\sigma]$ if and only if $\sigma_j=\tau_j$ for evey $j$).
	It is important to note that $\set{\sigma_j}_{j=1}^\ell$ partitions $[n]$. If $i\in[n]$ then $\matindex{\sigma(I)}_i=s_j$
	for some $j$ (which is distinct since the $s_j$s are distinct), and therefore $i\in\sigma_j$ for only that $j$.

	Since there are $k_j$ indexes which equal $s_j$, $\abs{\sigma_j}=k_j$. 

	We will construct a bijection:
	\[ f\colon S_{\sigma_1}\times\cdots\times S_{\sigma_\ell} \longrightarrow [\sigma] \]
	by:
	\[ f(\pi_1,\dots,\pi_\ell) = \tau \]
	Where $\tau$ is defined by:
	\[ \tau=\sigma\circ\pi \]
	If $i\in[n]$ then there exists a unique $j\in[\ell]$ such that $i\in\sigma_j$. Then we define:
	\[ \pi(i) = \pi_j(i) \]
	$\pi$ is a bijection since $\pi_j$ is, and therefore so is $\tau$.
	Notice that if $i\in\sigma_j$, then $\matindex{(\sigma\circ\pi)(I)}_i=\matindex{\sigma(I)}_{\pi(i)}$, and since
	$i\in\sigma_j$, then $\pi(i)=\pi_j(i)\in\sigma_j$, so
	\[ \matindex{(\sigma\circ\pi)(I)}_i=\matindex{\tau(I)}_{i}=s_j=\matindex{\sigma(I)}_i \]
	Which means that $\tau\in[\sigma]$, so $f$ is well-defined.

	This function corresponds to taking $\sigma(I)$ and permuting its indexes as to not mess up their elements.

	This is injective since if:
	\[ f(\pi_1,\dots,\pi_\ell) = f(\pi'_1,\dots,\pi'_\ell) \]
	Then:
	\[ \sigma\circ\pi = \sigma\circ\pi' \implies \pi = \pi' \]
	But that means that $\pi(i)=\pi'(i)$ for every $i$, and therefore for every $j$:
	\[ \pi_j(i) = \pi'_j(i) \]
	(For relevant $i$s), and therefore $\pi_j=\pi'_j$. So $f$ is injective.

	This is surjective. If $\tau\in [\sigma]$, then we need to find $\pi_j$s such that their corresponding $\pi$ is equal to
	$\sigma^{-1}\circ\tau$.

	Suppose $i\in\sigma_j$, we define:
	\[ \pi_j(i) = \sigma^{-1}\circ\tau(i) \]
	We need to show that this is well-defined. So we must show that $\sigma^{-1}\circ\tau(i)\in\sigma_j$.
	\[ \matindex{\sigma(I)}_{\sigma^{-1}\circ\tau(i)} = \matindex{I}_{\sigma\circ\sigma^{-1}\circ\tau(i)}
	 = \matindex{I}_{\tau(i)} = \matindex{\tau(I)}_i \]
	And since $[\tau]=[\sigma]$, this is equal to $\matindex{\sigma(I)}_i=s_j$.
	So:
	\[ \matindex{\sigma(I)}_{\sigma^{-1}\circ\tau(i)} = s_j \implies \sigma^{-1}\circ\tau(i)\in\sigma_j \]
	As required.

	And since for all relevant $i$: $\pi_j(i)=\sigma^{-1}\circ\tau(i)$, this means $\pi=\sigma^{-1}\circ\tau$, as required.

	So $f$ is a bijection. This means:
	\[ \abs{[\sigma]} = \abs{S_{\sigma_1}}\cdots\abs{S_{\sigma_\ell}} = k_1!\cdots k_\ell! \]

	By \ppref[lemma]{partCardLemma}, this the number of distinct permutations is:
	\[ \frac{n!}{k_1!\cdots! k_\ell!} \]
	As required.

\hfill$\qed$

\end{proof}

\newpage
\begin{defn*}

	We define the result of the previous lemma to be something called a \ppemph{multinomial coefficient}:
	\[ \binom{n}{k_1,\dots,k_\ell} = \frac{n!}{k_1!\cdots k_\ell!} \]
	If $\sum\limits_{i=1}^\ell k_i=n$.

\end{defn*}

A natural question to ask at this point is why is this called a \ppemph{multinomial} coefficient:
What is its relation with binomial coefficients?

\begin{prop*}[nomialRelationProposition]

	\[ \binom{n}{k} = \binom{n}{k,n-k} \]
	(The left side is a \ppemph{binomial coefficient} while the right side is a \ppemph{multinomial coefficient}.)

\end{prop*}

\begin{proof}

	Notice that $\binom{n}{k,n-k}$ represents the number of distinct tuples with $k$ of one element and $n-k$ of another.
	This is analogous, equivalent actually, to the number of ways to order $k$ of one element and $n-k$ of another.
	And as we discusses in the proof of \ppref[lemma]{binomialIntro}, this is equal to the cardinality of $\powsetof[k]{[n]}$.

	Therefore
	\[ \binom{n}{k} = \binom{n}{k,n-k} \]
	As required.

\hfill$\qed$

\end{proof}


\newpage
\begin{defn*}

	Given a tuple $I$ of length $n$ over a set of elements $S$, I define for every $s\in S$:
	\[ \#_s I\coloneqq\abs{\set{i\in[n]}[\matindex{I}_i=s]} \]
	Which is the number of occurrences of $s$ in the tuple $I$.

\end{defn*}

\begin{thrm*}[multinomialTheorem,The\ Multinomial\ Theorem]

	\[ \parens{\sum_{i=1}^\ell a_i}^n = \sum_{k_1+\cdots+k_\ell=n} \binom{n}{k_1,\dots,k_\ell}\cdot a_1^{k_1}\cdots a_\ell^{k_\ell} \]

	This is a generalization of the more famous \ppemph{Binomial Theorem}, which we will discuss soon.

\end{thrm*}

\begin{proof}

	We know:
	\[ \parens{\sum_{i=1}^\ell a_i}^n = \sum_{i_1=1}^\ell a_{i_1} \cdots\sum_{i_n=1}^\ell a_{i_n} =
	\sum_{i_1=1}^\ell\cdots\sum_{i_n=1}^\ell\parens{a_{i_1}\cdots a_{i_n}} \]
	Let $I$ be defined as the tuple:
	\[ I\coloneqq\tuple{i_1,\dots,i_n} \]
	We can rewrite the sum as:
	\[ \sum_{I\in[\ell]^n} \prod_{i\in I} a_i \]

	We can partition $[\ell]^n$ like so: for every $\set{k_i}_{i=1}^\ell$ such that $\sum\limits_{i=1}^\ell k_i$, create a
	subset of $[\ell]^n$ defined as follows:
	\[ \set{I\in[\ell]^n}[\forall j\in[\ell]: \#_j I = k_j] \]
	These sets are obviously disjoint, and their union is $[\ell]^n$ since for every $I\in[\ell]^n$ define $k_j=\#_j I$.

	Suppose $S$ is one of these sets characterized by $\set{k_i}_{i=1}^\ell$. Then for every $I\in S$:
	\[ \prod_{i\in I} a_i = a_1^{k_1}\cdots a_\ell^{k_\ell} \]
	Since there are $k_i$ instances of $i$ in $I$.

	And we know by the previous lemma that $\abs{S}=\binom{n}{k_1,\dots,k_\ell}$. So summing over $S$ gives:
	\[ \binom{n}{k_1,\dots,k_\ell}a_1^{k_1}\cdots a_\ell^{k_\ell} \]
	
	And since $S$ is characterized by and only by $\set{k_i}_{i=1}^\ell$, there are as many sets ($S$) in the partition as
	there are sequences $\set{k_i}_{i=1}^\ell$ such that $\sum\limits_{i=1}^\ell k_i = n$.

	This means:
	\[ \parens{\sum_{i=1}^\ell a_i}^n = \sum_{k_1+\cdots+k_\ell} \binom{n}{k_1,\dots,k_\ell}\cdot a_1^{k_1}\cdots a_{\ell}^{k_\ell} \]
	As required.

\hfill$\qed$

\end{proof}

\begin{thrm*}[binomialTheorem,The\ Binomial\ Theorem]

	\[ (a+b)^n = \sum_{k=0}^n\binom{n}{k} a^k\cdot b^{n-k} \]

\end{thrm*}

\begin{proof}

	By the \ppref{multinomialTheorem}, this is equal to:
	\[ (a+b)^n = \sum_{k_1+k_2=n}\binom{n}{k_1,k_2} a^{k_1}\cdot b^{k_2} \]
	But $k_1+k_2=n$ if and only if $k_2=n-k_1$. That is, we can construct a simple bijection from the set of $k_1,k_2$s to
	the set $\set{0,\dots,n}$ (by mapping $(k_1,k_2)$ to $k_1$).
	
	So instead we can sum over $\set{0,\dots,n}$:
	\[ (a+b)^n = \sum_{k=0}^n\binom{n}{k,n-k} a^k\cdot b^{n-k} \]
	And by \ppref[proposition]{nomialRelationProposition}, this is equal to:
	\[ (a+b)^n = \sum_{k=0}^n\binom{n}{k} a^k\cdot b^{n-k} \]
	As required.

\hfill$\qed$

\end{proof}


